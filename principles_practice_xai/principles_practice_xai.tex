\documentclass[11pt,dvipsnames,usenames,aspectratio=169]{beamer}  % Add handout to options to disable overlays

% For more themes, color themes and font themes, see:
% http://deic.uab.es/~iblanes/beamer_gallery/index_by_theme.html
%
\mode<presentation>
{%
  \usetheme{CambridgeUS}    % or try default, Darmstadt, Warsaw, ...
  \usecolortheme{whale}     % or try albatross, beaver, crane, ...
  \usefonttheme{serif}          % or try default, structurebold, ...
  % \usefonttheme[onlymath]{serif}
  % \setbeamertemplate{navigation symbols}{}
  % \setbeamercovered{transparent}

  \setbeamercolor{title}{fg=white}
  \setbeamerfont{title}{series=\bfseries}
  \setbeamercolor{frametitle}{fg=black}
  \setbeamerfont{frametitle}{series=\bfseries}

  \setbeamercolor{section in head/foot}{fg=white}
  \setbeamerfont{section in head/foot}{series=\bfseries}
  \setbeamercolor{subsection in head/foot}{fg=white}
  \setbeamerfont{subsection in head/foot}{series=\bfseries}
  \setbeamercolor{author in head/foot}{fg=white}
  \setbeamerfont{author in head/foot}{series=\bfseries}
  \setbeamercolor{title in head/foot}{fg=white}
  \setbeamerfont{title in head/foot}{series=\bfseries}

  \setbeamercolor{block title}{use=structure,fg=white,bg=title in head/foot.bg}
  \setbeamerfont{block title}{series=\bfseries}
  \setbeamercolor{block body}{use=structure,fg=black,bg=black!1!white}
}

% Support graying out frame elements
\newcommand{\FrameOpague}{\setbeamercovered{again covered={\opaqueness<1->{40}}}}
% Transition slide
\newcommand{\transitionFrame}[1]{%
{%
  \begin{frame}[plain,noframenumbering]{}{} % the plain option removes the sidebar and header from the title page
    \setbeamertemplate{final page}[text]{\Large \textbf{#1}}
    \usebeamertemplate{final page}
  \end{frame}}
}

% Imported via UltiSnips
\usepackage{graphicx} % Loading the package
\graphicspath{{img/}} % Setting the default folder containing the graphics

\usepackage[numbers]{natbib}
\usepackage{appendixnumberbeamer}
\usepackage{hyperref}

% Here's where the presentation starts, with the info for the title slide
\title[Principles \& Practice of Explainable ML]{Principles and Practice of Explainable Machine Learning}
\author[Belle \& Papantonis]{%
  {Vaishak Belle}\inst{1}\inst{2}
  \and
  {Ioannis Papantonis}\inst{1}
}

\institute[Univ.\ Edinburgh]{%
  \textsuperscript{1}\textbf{University of Edinburgh}
  \and
  \textsuperscript{2}\textbf{Alan Turing Institute}
}
\date{October~8, 2020}

% Imported via UltiSnips
\input{global_macros}
\newcommand{\dec}{f}
\newcommand{\domainX}{X}
\newcommand{\domainY}{Y}

\newcommand{\X}{x}

\newcommand{\dist}{\mathcal{D}}

\newcommand{\Rule}{A}
\newcommand{\setRule}{\mathcal{A}}

\newcommand{\Prec}[1]{{\text{prec}(#1)}}
\newcommand{\minPrec}{\tau}

\newcommand{\cov}[1]{{\text{cov}(#1)}}
% \newcommand{\minPrec}{\tau}


% Enable (uncolored) cross-reference hyperlinks
% Should always be last package loaded.
% See: https://tex.stackexchange.com/questions/103123/links-do-not-lead-to-right-pages
\usepackage[colorlinks=false]{hyperref}

\begin{document}

\begin{frame}
  \titlepage
\end{frame}

\begin{frame}{Let's Start with a Case Study\ldots}
  \noindent
  Before examining the problem from a technical perspective, let's think through a \textit{real-world} scenario
  \begin{itemize}
    \item Help us identify salient ideas
    \item Allow us to construct a working vocabulary
  \end{itemize}

  \vspace{18pt}
  \onslide<2->{
    \noindent
    \textbf{Note}: A case study appears at the end of paper, but I think the following will yield a more fruitful discussion.
  }
\end{frame}

\begin{frame}{Fair Criminal Sentencing}
  \begin{itemize}[<+->]
    \setlength{\itemsep}{22pt}
    \onslide<+->{\item I get this is a bit heavy topic\ldots}

    \onslide<+->{\item \textbf{Sentence}: Punishment (e.g.,~fine, probation, imprisonment, restituion) assigned in relation to a specific criminal act.}

    \onslide<+->{\item \green{\textbf{Accurate}}/appropriate sentencing is \textbf{critical} for a healthy justice system.}

    \onslide<+->{%
      \item \textbf{Key Attributes of an Appropriate Sentencing Apparatus}
      \begin{itemize}
        \setlength{\itemsep}{3pt}
        \item \textit{Consistent}: Similar crimes/circumstances should yield similar punishments.
        \item \textit{Understandable}: Clarity \textit{why} the selected sentence is appropriate
        \item \textit{Transparent}: Knowledge of \textit{how} the sentence was selected sentence
        \item \textit{Sufficient}: Ensures justice for the victims and keeps the public safe
      \end{itemize}
    }
  \end{itemize}
\end{frame}

\begin{frame}{Problem with Sentencing Today}
  \onslide<+->{\textbf{\blue{Major Problem}}: Human juries/judges badly affected by bias\ldots}

  \onslide<+->{%
    \begin{itemize}
      \setlength{\itemsep}{22pt}
      \item \textbf{Racial Bias}: An African-American defendant is 5.9x as likely to be imprisoned as a white defendant. Hispanics are 3.1x as likely.~\citep{SentencingProject}

      \item \textbf{Gender Bias}: Men receive 63\% longer sentences on average than women do. Women are twice as likely to avoid incarceration if convicted.~\citep{Starr:2014}

      \item \textbf{Wealth Effects}: Poor defendants more likely to plead guilty since they cannot afford bail and lack a safety net to afford remaining in prison until trial
    \end{itemize}
  }
\end{frame}

\begin{frame}{Here's an Idea\ldots}
  \onslide<+->{Take the \textit{humans} out of sentencing
    \begin{itemize}
      \item Make it entirely ``data driven''
    \end{itemize}
  }

  \vspace{22pt}
  \onslide<+->{%
    Use machine learning to estimate the \green{recidivism rate}. \textbf{Three types of recidivism}:
    \begin{itemize}
      \item \textit{Pretrial recidivism}
      \item \textit{General recidivism}
      \item \textit{Pretrial recidivism}
    \end{itemize}
  }

  \vspace{22pt}
  \onslide<+->{\blue{\textbf{COMPAS}}: \underline{C}orrectional \underline{O}ffender \underline{M}anagement \underline{P}rofiling for \underline{A}lternative \underline{S}anctions
    \begin{itemize}
      \item Released: 2009
      \item \green{\textbf{Proprietary}} (Developed by private company Northpointe)
    \end{itemize}
  }
\end{frame}

\begin{frame}{You May See Where This is Going\ldots}
  \begin{center}
    \onslide<2->{
      \includegraphics[scale=0.55]{propublica_compas_small.png}
    }
  \end{center}
\end{frame}

\begin{frame}[allowframebreaks]{Bibliography}
  {\tiny
    \frametitle{References}
    \bibliography{bib/ref.bib}
    \bibliographystyle{unsrtnat}
  }
\end{frame}

\begin{frame}{Some of the Highlights (or Lowlights)\ldots}
  \noindent
  \onslide<+->{\textbf{Source}:~\citet{Angwin:2016} (7~years after COMPAS' initial release)}

  \begin{itemize}[<+->]
    \setlength{\itemsep}{12pt}
    \item ``blacks are almost twice as likely as whites to be labeled a higher risk but not actually re-offend''

    \item ``makes the opposite mistake among whites: They are much more likely than blacks to be labeled lower-risk but go on to commit other crimes.''

    \item Only 20\% of people predicted to commit violent crimes actually went on to do so.

    \item ``More \green{accurate} than a human''~\citep{Dressel:2018}
      \begin{itemize}[<+->]
        \item Individual with \textit{No Expertise} in Criminal Justice: 63\% accuracy
        \item COMPAS: 65\% accuracy
      \end{itemize}
  \end{itemize}

  \onslide<+->{
    \begin{block}{Takeaway}
      Explainable machine learning may have identified the model's implicit bias.
    \end{block}
  }
\end{frame}

\begin{frame}{Summary}
  \onslide<+->{
    \blue{\textbf{Key Themes of Practical Machine Learning}}:
      \begin{itemize}
        \setlength{\itemsep}{6pt}
        \item Proprietary IP vs.\ Transparency
        \item Model accuracy
        \item (Racial) bias
        \item Human understandability
        \item Importance of individual features (e.g.,~gender)
      \end{itemize}
  }

  \vspace{16pt}
  \onslide<+->{%
    \green{\textbf{Discussion Questions}}:
    \begin{itemize}
      \item What other high-level themes do you see in this example?
      \item Is there such a thing as a free lunch? Can I have it all?
    \end{itemize}
  }
\end{frame}

\begin{frame}{Types of Explanations}{}
  {\small
    \begin{itemize}[<+->]
      \setlength{\itemsep}{15pt}
      \item \textbf{\green{Text Explanation}}: Using symbols (e.g.,~natural language text, or propositional sumbols) to explain the model's behavior
      \item \textbf{\green{Visual Explanation}}: Generate visualization that facilitate understanding, e.g.,~heat map
      \item \textbf{\green{Local Explanation}}: Explain how a model operates in a certain area of interest, e.g.,~explain the prediction for a specific instance
      \item \textbf{\green{Explanations by Example}}: Extract \textit{representative instances} from the training dataset
      \item \textbf{\green{Explanation by Simplification}}: Approximate an \textit{opaque} model using a simpler, easier-to-interpret one
      \item \textbf{\green{}}:
    \end{itemize}
  }
\end{frame}

\begin{frame}{Explanations by Example}{}
  \begin{columns}
    \begin{column}{0.45\textwidth}
      \begin{itemize}
        \setlength{\itemsep}{22pt}
        \item Common approach in human learning via \green{prototypical examples}
        \item \blue{\textbf{Example}}: \href{https://quickdraw.withgoogle.com/}{Google Quick, Draw!}~\citep{Cai:2019}
          \begin{itemize}
          \setlength{\itemsep}{8pt}
            \item Like charades or the old TV~show ``\href{https://en.wikipedia.org/wiki/Win,_Lose_or_Draw}{Win, Lose, or Draw}''
            \item Given a prompt and the user draws it
            \item AI tries to guess the picture
          \end{itemize}
      \end{itemize}
    \end{column}
    \begin{column}{0.1\textwidth}
    \end{column}
    \begin{column}{0.45\textwidth}
      \onslide<2->{%
        \includegraphics[scale=0.20]{avocado_quick_draw.png}
      }
    \end{column}
  \end{columns}
\end{frame}

\begin{frame}{Local Explanation}
  \blue{\textbf{Obvious Question}}: Why was this \textit{particular instance} mispredicted?

  \vspace{16pt}
  \begin{center}
    \onslide<2->{
          \includegraphics[scale=0.264]{avocado_alternates_quick_draw.png}
    }
  \end{center}
\end{frame}
\end{document}

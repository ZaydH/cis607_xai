\newcommand{\etal}{~et~al.}

% Used for including standalone docs
\usepackage{standalone}

% Imported via UltiSnips
% Unbreakable dash:
%  Words hyphened with these dashes can also be broken at other positions than the dash
%    \-/ hyphen
%    \-- en-dash
%    \--- em-dash
%    extdash unbreakable dashes
%
%  No line breaks possible at the hyphen
%    \=/ hyphen
%    \== en-dash
%    \=== em-dash
\usepackage[shortcuts]{extdash}

% Imported via UltiSnips
% \usepackage{xcolor}
\newcommand{\colortext}[2]{{\color{#1} #2}}
\newcommand{\red}[1]{\colortext{red}{#1}}
\newcommand{\blue}[1]{\colortext{blue}{#1}}
\newcommand{\green}[1]{\colortext{ForestGreen}{#1}}

% Imported via UltiSnips
\usepackage{amsmath}
\DeclareMathOperator*{\argmax}{arg\,max}
\DeclareMathOperator*{\argmin}{arg\,min}
\usepackage{amsfonts}  % Used for \mathbb and \mathcal
\usepackage{amssymb}

% Imported via UltiSnips
\usepackage{mathtools} % for "\DeclarePairedDelimiter" macro
% \swapifbranches changes unstarred paired delimiters to starred and
% vice versa.  This means by default, paired delimiters have the star.
\usepackage{etoolbox}
\newcommand\swapifbranches[3]{#1{#3}{#2}}
\makeatletter
\MHInternalSyntaxOn
\patchcmd{\DeclarePairedDelimiter}{\@ifstar}{\swapifbranches\@ifstar}{}{}
\MHInternalSyntaxOff
\makeatother
% Place after swap to ensure swap star
\DeclarePairedDelimiter{\sbrack}{\lbrack}{\rbrack}
\DeclarePairedDelimiter{\floor}{\lfloor}{\rfloor}
\DeclarePairedDelimiter{\ceil}{\lceil}{\rceil}
\DeclarePairedDelimiter{\abs}{\lvert}{\rvert}
\DeclarePairedDelimiter{\round}{\lfloor}{\rceil}
\DeclarePairedDelimiter{\norm}{\lVert}{\rVert}
\usepackage{bm}
\DeclarePairedDelimiterX\set[1]\lbrace\rbrace{#1}
\DeclarePairedDelimiterX\setbuild[2]\lbrace\rbrace{#1 \bm: #2}
\newcommand{\setint}[1]{{\sbrack{#1}}}
\newcommand{\func}[3]{{#1:#2\rightarrow#3}}
% \newcommand{\defeq}{\stackrel{\mathclap{\mbox{\tiny def}}}{=}}
\newcommand{\defeq}{\coloneqq}
\newcommand{\fedeq}{\eqqcolon}
\newcommand{\expect}[1]{\mathbb{E}\sbrack{#1}}
% Expectation with the subscript defining the distribution
\newcommand{\expectS}[2]{\mathbb{E}_{#1}\sbrack{#2}}

% Allow numbering in align*
\newcommand{\numberthis}{\addtocounter{equation}{1}\tag{\theequation}}

\newcommand{\ints}{\mathbb{Z}}
\newcommand{\intsNN}{\mathbb{Z}_{+}}
\newcommand{\nats}{\mathbb{N}}
\newcommand{\real}{\mathbb{R}}
\newcommand{\realnn}{\real_{{\geq}0}}  % Set of non-negative real numbers

\newcommand{\iidsim}{\stackrel{\mathclap{\mbox{\tiny \textnormal{i.i.d.}}}}{\sim}}

\newcommand{\normaldist}[2]{{\mathcal{N}\mathopen{}\left(#1,#2\right)\mathclose{}}}

% Imported via UltiSnips
\usepackage{array}  % Provides a way add a \centering command to a p-column
\usepackage{arydshln}  % Introduces hdashline & cdashline
\usepackage{bigdelim}
\usepackage{booktabs}
\usepackage{multirow}
\usepackage{makecell}  % Needed for multirowcell

% % Imported via UltiSnips
\usepackage{amsthm}
% \newtheorem{theorem}{Theorem}
% \newtheorem{corollary}{Corollary}[theorem]  % Corollary number derives from theorem
% \newtheorem{lemma}[theorem]{Lemma}  % Lemma and theorem share same counter
% \newtheorem{claim}[theorem]{Claim}  % Same numbering as lemma and theorem
% \newtheorem*{remark}{Remark}
% \newtheorem*{note}{Note}
% \newtheoremstyle{definition}  % <name>
% {3pt}   % <Space above>
% {3pt}   % <Space below>
% % {\itshape}     % <Body font>
% {\normalfont}   % <Body font>
% {}      % <Indent amount>
% {\bfseries} % <Theorem head font>
% {:}     % <Punctuation after theorem head>
% {.5em}  % <Space after theorem head>
% {}      % <Theorem head spec (can be left empty, meaning `normal')>
% \theoremstyle{definition}
% \newtheorem{definition}{Def.}[section]

% Imported via UltiSnips
\usepackage{tikz}
\usetikzlibrary{arrows,decorations.markings,shadows,positioning,calc,backgrounds,shapes}

\usepackage{pgfplots}
\pgfplotsset{compat=1.13}
\usepackage{pgfplotstable}
% \usepackage{subcaption}  % Cannot be used with subfigure

% Handle empty parameters
\usepackage{xifthen}
\newcommand{\ifempty}[3]{%
  \ifthenelse{\isempty{#1}}{#2}{#3}%
}

\documentclass[11pt,dvipsnames,usenames,aspectratio=169]{beamer}  % Add handout to options to disable overlays

% For more themes, color themes and font themes, see:
% http://deic.uab.es/~iblanes/beamer_gallery/index_by_theme.html
%
\mode<presentation>
{%
  \usetheme{CambridgeUS}    % or try default, Darmstadt, Warsaw, ...
  \usecolortheme{whale}     % or try albatross, beaver, crane, ...
  \usefonttheme{serif}          % or try default, structurebold, ...
  % \usefonttheme[onlymath]{serif}
  % \setbeamertemplate{navigation symbols}{}
  % \setbeamercovered{transparent}

  \setbeamercolor{title}{fg=white}
  \setbeamerfont{title}{series=\bfseries}
  \setbeamercolor{frametitle}{fg=black}
  \setbeamerfont{frametitle}{series=\bfseries}

  \setbeamercolor{section in head/foot}{fg=white}
  \setbeamerfont{section in head/foot}{series=\bfseries}
  \setbeamercolor{subsection in head/foot}{fg=white}
  \setbeamerfont{subsection in head/foot}{series=\bfseries}
  \setbeamercolor{author in head/foot}{fg=white}
  \setbeamerfont{author in head/foot}{series=\bfseries}
  \setbeamercolor{title in head/foot}{fg=white}
  \setbeamerfont{title in head/foot}{series=\bfseries}

  \setbeamercolor{block title}{use=structure,fg=white,bg=title in head/foot.bg}
  \setbeamerfont{block title}{series=\bfseries}
  \setbeamercolor{block body}{use=structure,fg=black,bg=black!1!white}
}

% Support graying out frame elements
\newcommand{\FrameOpaque}{\setbeamercovered{again covered={\opaqueness<1->{40}}}}
% Transition slide
\newcommand{\transitionFrame}[1]{%
{%
  \begin{frame}[plain,noframenumbering]{}{} % the plain option removes the sidebar and header from the title page
    \setbeamertemplate{final page}[text]{\Large \textbf{#1}}
    \usebeamertemplate{final page}
  \end{frame}}
}

\usepackage{fontawesome}  % Used for defining the external hyperlink

% Imported via UltiSnips
\usepackage{graphicx} % Loading the package
\graphicspath{{img/}} % Setting the default folder containing the graphics

\usepackage[numbers]{natbib}
\usepackage{appendixnumberbeamer}

% \hypersetup{colorlinks=true,allcolors=blue}

% Here's where the presentation starts, with the info for the title slide
\title[Anchors]{Anchors: High-Precision Model-Agnostic Explanations}
\author[Belle \& Papantonis]{%
  {Marco Tulio Ribeiro}\inst{1}
  \and
  {Sameer Singh}\inst{2}
  \and
  {Carlos Guestrin}\inst{1}
}

\institute[UW \& UCI]{%
  \textsuperscript{1}\textbf{University of Washington}
  \and
  \textsuperscript{2}\textbf{University of California, Irvine}
}
\date{November~12, 2020}

% Imported via UltiSnips
\input{global_macros}
\newcommand{\dec}{f}
\newcommand{\domainX}{X}
\newcommand{\domainY}{Y}

\newcommand{\X}{x}

\newcommand{\dist}{\mathcal{D}}

\newcommand{\Rule}{A}
\newcommand{\setRule}{\mathcal{A}}

\newcommand{\Prec}[1]{{\text{prec}(#1)}}
\newcommand{\minPrec}{\tau}

\newcommand{\cov}[1]{{\text{cov}(#1)}}
% \newcommand{\minPrec}{\tau}


% Enable (uncolored) cross-reference hyperlinks
% Should always be last package loaded.
% See: https://tex.stackexchange.com/questions/103123/links-do-not-lead-to-right-pages
\usepackage{hyperref}
\let\oldhref\href
\renewcommand{\href}[2]{\oldhref{#1}{#2\hspace{0.05cm}{\scriptsize\faExternalLink}}}

\begin{document}

\begin{frame}
  \titlepage
\end{frame}

\section{On Rules of Thumb}

\begin{frame}[noframenumbering]{``If silence be good for the wise, how much better for fools.''}
  It takes quite a fool to quote one's self\ldots

  \vspace{25pt}
  \onslide<2->{You're in luck that I am just such a fool...}
\end{frame}

\begin{frame}{SAT'2018 Paper}
  Roughly speaking, the end result is a sampler for which one can largely \\ use the following \green{rule of thumb}:

  \vspace{20pt}
  \begin{center}
    Generating 1,000 perfectly uniform models takes \\
    about 10 times as long as it takes to count the models.
  \end{center}
\end{frame}

\begin{frame}{Week 2 Presentation}
  \begin{center}
    \includegraphics[scale=0.15]{rule_of_thumb_pres.png}
  \end{center}
\end{frame}

\begin{frame}{Week 5 Discussion on Canvas}
  \textbf{Paper}: ``Manipulating and Measuring Model Interpretability''~\citep{PoursabziSangdeh:2018} \\
  \textbf{Date}: October~29, 2020

  \vspace{12pt}
  ``I would summarize a \green{rule of thumb} for this as for most cases, combining `man and machine' was a bad idea.''
\end{frame}

\begin{frame}{What is a Rule of Thumb \& Why am I Obsessed with Them?}
  \textbf{Standard Definition}:
  \begin{itemize}
    \item Broadly but not strictly accurate
    \item Based on experience or practice rather than theory
  \end{itemize}

  \vspace{20pt}
    \onslide<2->{\textbf{Wikipedia Criteria}: ``easily learned and easily applied procedure''
    \begin{itemize}
      \item You might say ``short and sweet''
    \end{itemize}
  }
\end{frame}

\section{On Anchors}

\begin{frame}{LIME is a Bit \textit{Sour}}
  \blue{\textbf{LIME}}: Local Interpretable Model-Agnostic Explanations~\citep{Ribeiro:2016}
  \begin{itemize}
    \item \textbf{Authors}: Marco Tulio Ribeiro, Sameer Singh, \& Carlos Guestrin (sound familiar?)
  \end{itemize}

  \vspace{20pt}
  \textbf{\green{Similarity with Rules of Thumb}}
  \begin{itemize}
    \item Model-Agnostic $\rightarrow$ Broadly Applicable
  \end{itemize}
\end{frame}

\begin{frame}[noframenumbering]{Table of Contents}
  \tableofcontents
\end{frame}

\begin{frame}[noframenumbering,allowframebreaks]{Bibliography}{}
  {\tiny
    \frametitle{References}
    \bibliography{bib/ref.bib}
    \bibliographystyle{unsrtnat}
  }
\end{frame}
\end{document}
